\section { Input Data Examples}
\label{readdata}

\subsection { Read PDG Particle Data }
\begin{verbatim}
// $Id: readdata.tex,v 1.2 2003/08/26 20:49:21 garren Exp $
// ----------------------------------------------------------------------
// TestReadPDG.cc
//
// read PDG table and write it out
//
// ----------------------------------------------------------------------

#include <fstream>

#include "HepPDT/TableBuilder.hh"
#include "HepPDT/ParticleDataTable.hh"
#include "HepPDT/DecayModel.hh"

int main()
{
    const char pdgfile[] = "../data/PDG_mass_width_2002.mc";
    const char outfile[] = "PDfile";
    // open input file
    std::ifstream pdfile( pdgfile );
    if( !pdfile ) { 
      std::cerr << "cannot open " << pdgfile << std::endl;
      exit(-1);
    }
    // construct empty PDT
    HepPDT::ParticleDataTable datacol( "PDG Table" );
    {
        // Construct table builder
        HepPDT::TableBuilder  tb(datacol);
	// read the input - put as many here as you want
	if( !addPDGParticles( pdfile, tb ) ) { std::cout << "error reading PDG file " << std::endl; }
    }	// the tb destructor fills datacol
    std::ofstream wpdfile( outfile );
    if( !wpdfile ) { 
      std::cerr << "cannot open " << outfile << std::endl;
      exit(-1);
    }
    datacol.writeParticleData(wpdfile);
}
\end{verbatim}

\subsection { Read EvtGen Particle Data }
\begin{verbatim}
// $Id: readdata.tex,v 1.2 2003/08/26 20:49:21 garren Exp $
// ----------------------------------------------------------------------
// TestReadEvtGen.cc
//
// read EvtGen table and write it out
//
// ----------------------------------------------------------------------

#include <fstream>

#include "HepPDT/TableBuilder.hh"
#include "HepPDT/ParticleDataTable.hh"

int main()
{
    const char infile1[] = "data/pdt.table";
    const char infile2[] = "data/DECAY.EvtGen.DEC";
    const char outfile[] = "PDfile3";
    // open input files
    std::ifstream pdfile1( infile1 );
    if( !pdfile1 ) { 
      std::cerr << "cannot open " << infile1 << std::endl;
      exit(-1);
    }
    // construct empty PDT
    std::ifstream pdfile2( infile2 );
    if( !pdfile2 ) { 
      std::cerr << "cannot open " << infile2 << std::endl;
      exit(-1);
    }
    HepPDT::ParticleDataTable datacol( "EvtGen Table" );
    {
        // Construct table builder
        HepPDT::TableBuilder  tb(datacol);
	// read the input - put as many here as you want
        if( !addEvtGenParticles( pdfile1, tb ) ) { std::cout << "error reading EvtGen pdt file " << std::endl; }
        if( !addEvtGenParticles( pdfile2, tb ) ) { std::cout << "error reading EvtGen decay file " << std::endl; }
    }	// the tb destructor fills datacol
    std::ofstream wfile( outfile );
    if( !wfile ) { 
      std::cerr << "cannot open " << outfile << std::endl;
      exit(-1);
    }
    datacol.writeParticleData(wfile);
}
\end{verbatim}

\subsection { Read Pythia Particle Data }
\begin{verbatim}
// $Id: readdata.tex,v 1.2 2003/08/26 20:49:21 garren Exp $
// ----------------------------------------------------------------------
// TestReadPythia.cc
//
// read Pythia table and write it out
//
// ----------------------------------------------------------------------

#include <fstream>

#include "HepPDT/TableBuilder.hh"
#include "HepPDT/ParticleDataTable.hh"
#include "HepPDT/TempParticleData.hh"

int main()
{
    const char infile[] = "data/pythia.tbl";
    const char outfile[] = "PDfile2";
    // open input file
    std::ifstream pdfile( infile );
    if( !pdfile ) { 
      std::cerr << "cannot open " << infile << std::endl;
      exit(-1);
    }
    // construct empty PDT
    HepPDT::ParticleDataTable datacol( "Pythia Table" );
    {
        // Construct table builder
        HepPDT::TableBuilder  tb(datacol);
	// read the input - put as many here as you want
        if( !addPythiaParticles( pdfile, tb ) ) 
	      { std::cout << "error reading pythia file " << std::endl; }
    }	// the tb destructor fills datacol
    std::ofstream wpdfile( outfile );
    if( !wpdfile ) { 
      std::cerr << "cannot open " << outfile << std::endl;
      exit(-1);
    }
    datacol.writeParticleData(wpdfile);
}
\end{verbatim}

\subsection { Read QQ Particle Data }
\begin{verbatim}
// $Id: readdata.tex,v 1.2 2003/08/26 20:49:21 garren Exp $
// ----------------------------------------------------------------------
// TestReadQQ.cc
//
// read QQ table and write it out
//
// ----------------------------------------------------------------------

#include <fstream>

#include "HepPDT/QQDecayTable.hh"
#include "HepPDT/PDGtoQQTable.hh"
#include "HepPDT/TableBuilder.hh"
#include "HepPDT/ParticleDataTable.hh"
#include "HepPDT/TempParticleData.hh"

int main()
{
    const char infile[] = "data/decay.dec";
    const char outfile[] = "PDfileQQ";
    // open input file
    std::ifstream pdfile( infile );
    if( !pdfile ) { 
      std::cerr << "cannot open " << infile << std::endl;
      exit(-1);
    }
    // read decay.dec 
    HepPDT::QQDecayTable qdt( pdfile );
    // create the translation table
    HepPDT::PDGtoQQTable::instance()->buildTable( qdt );
    // construct empty PDT
    HepPDT::ParticleDataTable datacol( "QQ Table" );
    {
        // Construct table builder
        HepPDT::TableBuilder  tb(datacol);
	// read the input - put as many here as you want
        if( !addQQParticles( qdt, tb ) ) 
	     { std::cout << "error reading QQ table file " << std::endl; }
    }	// the tb destructor fills the PDT
    std::ofstream wpdfile( outfile );
    if( !wpdfile ) { 
      std::cerr << "cannot open " << outfile << std::endl;
      exit(-1);
    }
    // write the tranlations
    HepPDT::PDGtoQQTable::instance()->writeTranslations( wpdfile );
    // write the particle and decay info
    datacol.writeParticleData( wpdfile );
}
\end{verbatim}

\vfill\eject
